\documentclass[dvips,12pt]{article}

% Any percent sign marks a comment to the end of the line

% Every latex document starts with a documentclass declaration like this
% The option dvips allows for graphics, 12pt is the font size, and article
%   is the style

\usepackage[letterpaper, margin=1in]{geometry}

\usepackage[pdftex]{graphicx}
\usepackage{tabularx}
\usepackage{multirow}
\usepackage{fancyhdr}

\usepackage[acronyms]{glossaries}
\newacronym{NFCS}{NFCS}{nuclear fuel cycle simulator}
\newacronym{TRL}{TRL}{technology readiness level}
\newacronym{FCO}{FCO}{Fuel Cycle Options}
\newacronym{USDOE}{USDOE}{U.S. Department of Energy}
\newacronym{VnV}{V\&V}{verification and validation}
\newacronym{FCCI}{FCCI}{Fuel Cycle Confidence Initiative}
\newacronym{CHTC}{CHTC}{Center for High-Throughput Computing}
\newacronym{ACI}{ACI}{Advanced Computing Initiative}
\newacronym{OSG}{OSG}{Open Science Grid}
\newacronym{HTC}{HTC}{high-throughput computing}
\newacronym{HPC}{HPC}{high-performance computing}


\pagestyle{fancy}
\fancyhf{}
\chead{\includegraphics[width=2in]{neup_logo.png}}
\renewcommand{\headrulewidth}{0pt}

\newcommand{\ID}{15645}
\newcommand{\mytitle}{Developing an Approach to Verification and Validation\\ for Nuclear Fuel Cycle Simulation Tools}
\newcommand{\workscope}{MS-FC-1}
\newcommand{\timeframe}{10/01/2018 - 09/30/2021}


\newcommand{\unc}[1]
{ \delta #1 }

\newcommand{\uncsq}[1]
{ \left(\unc{#1}\right)^2 }

\newcommand{\uncratio}[1]
{ \left(\frac{\unc{#1}}{#1}\right) }

\newcommand{\uncratiosq}[1]
{ \uncratio{#1}^2 }

\newcommand{\uncvector}[1]
{ \left[ \begin{array}{c} #1 \\ \delta #1 \end{array} \right] }

\newcommand{\comment}[1]
{{\bfseries \color{red} #1}}


%----------------------------------------------------------------------------------------
%	DOCUMENT INFORMATION
%----------------------------------------------------------------------------------------
\begin{document}
\vspace*{0.5in}

\noindent\rule{\textwidth}{1pt}\

\vspace*{0.2in}

\begin{centering}
  \textbf{\large \mytitle}\\
\end{centering}

\vspace*{0.25in}

\noindent
\begin{tabularx}{\textwidth}{rlrl}
   \textbf{PI:} & Paul P.H. Wilson, U. Wisconsin-Madison \\
   \textbf{Program:} & Fuel Cycle
\end{tabularx}



\vspace{0.2in}
\noindent\rule{\textwidth}{1pt}\\

\noindent\textbf{Abstract:} 

The objective of this project is to develop an open and collaborative approach
for establishing increased confidence in \gls{NFCS} tools. This approach will
directly address verification by publishing jointly-developed problems with
reference solutions.  It will indirectly address validation through a growing
set of publicly available solutions to those problems contributed by
developers of such tools.  Together, this approach will allow the nuclear
fuel cycle community to develop a more nuanced understanding of the results
produced by their tools and, in so doing, be better prepared to express
confidence in those results when communicating results both within and without
that community.

This kind of comparison will also build knowledge and understanding about the
differences between the various algorithms and models used by the many
\gls{NFCS} tools. In particular, it is important to understand the
circumstances under which the choice of algorithm and/or model influences the
conclusions that one might draw from such a simulation. Such knowledge could
not only help fuel cycle analysts to perform more pertinent and relevant
studies, but ultimately to convince stakeholders to rely more on nuclear fuel
cycle analysis in the decision making processes relative to nuclear
technology.

The development of this capability in an open and collaborative fashion will
support innovation in this space, as it makes it possible for new entrants to
immediately begin testing their tools against established solutions without
waiting for a coordinated benchmarking effort.  Such coordinated efforts have
occurred in the past, but their results are generally not published in a way
that invites comparison by future analysts who were not part of the initial
exercise.



\end{document}
