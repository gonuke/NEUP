\documentclass[dvips,12pt]{article}

% Any percent sign marks a comment to the end of the line

% Every latex document starts with a documentclass declaration like this
% The option dvips allows for graphics, 12pt is the font size, and article
%   is the style

\usepackage[letterpaper, margin=1in]{geometry}

\usepackage[pdftex]{graphicx}
\usepackage{tabularx}

\usepackage{fancyhdr}
\pagestyle{fancy}
\fancyhf{}
\chead{\includegraphics[width=2in]{neup_logo.png}}
\renewcommand{\headrulewidth}{0pt}

\newcommand{\unc}[1]
{ \delta #1 }

\newcommand{\uncsq}[1]
{ \left(\unc{#1}\right)^2 }

\newcommand{\uncratio}[1]
{ \left(\frac{\unc{#1}}{#1}\right) }

\newcommand{\uncratiosq}[1]
{ \uncratio{#1}^2 }

\newcommand{\uncvector}[1]
{ \left[ \begin{array}{c} #1 \\ \delta #1 \end{array} \right] }

\newcommand{\comment}[1]
{{\bfseries \color{red} #1}}


%----------------------------------------------------------------------------------------
%	DOCUMENT INFORMATION
%----------------------------------------------------------------------------------------
\begin{document}
\vspace*{0.5in}
\begin{centering}
  \textbf{\large Error and Uncertainty Propagation for Fuel Cycle Calculations}\\
\end{centering}

\noindent\rule{\textwidth}{0.4pt}\\

\noindent
\begin{tabular}{rll}
\textbf{PI:} & Paul Wilson & University of Wisconsin-Madison\\
\textbf{Collaborators:} & Baptiste Mouginot & University of Wisconsin-Madison\\
\textbf{Technical Workscope:} & Fuel Cycle  & (FC-5.1b)\\
\end{tabular}

\vspace{0.2in}
\noindent\rule{\textwidth}{0.4pt}\\

\noindent\textbf{Abstract:} 

\end{document}
