\documentclass[dvips,12pt]{article}

% Any percent sign marks a comment to the end of the line

% Every latex document starts with a documentclass declaration like this
% The option dvips allows for graphics, 12pt is the font size, and article
%   is the style

\usepackage[letterpaper, margin=1in]{geometry}

\usepackage[pdftex]{graphicx}
\usepackage{tabularx}
\usepackage{multirow}
\usepackage{fancyhdr}
\pagestyle{fancy}
\fancyhf{}
\chead{\includegraphics[width=2in]{neup_logo.png}}
\renewcommand{\headrulewidth}{0pt}

\newcommand{\unc}[1]
{ \delta #1 }

\newcommand{\uncsq}[1]
{ \left(\unc{#1}\right)^2 }

\newcommand{\uncratio}[1]
{ \left(\frac{\unc{#1}}{#1}\right) }

\newcommand{\uncratiosq}[1]
{ \uncratio{#1}^2 }

\newcommand{\uncvector}[1]
{ \left[ \begin{array}{c} #1 \\ \delta #1 \end{array} \right] }

\newcommand{\comment}[1]
{{\bfseries \color{red} #1}}


%----------------------------------------------------------------------------------------
%	DOCUMENT INFORMATION
%----------------------------------------------------------------------------------------
\begin{document}
\vspace*{0.5in}

\noindent\rule{\textwidth}{1pt}\

\vspace*{0.2in}

\begin{centering}
  \textbf{\large Error and Uncertainty Propagation for Fuel Cycle Calculations}\\
\end{centering}

\vspace*{0.25in}

\noindent
\begin{tabularx}{\textwidth}{rlrl}
   \textbf{PI:} & Paul P.H. Wilson, & 
         \textbf{Collaborators:} & Baptiste Mouginot, \\
                & U. Wisconsin-Madison & 
         & U. Wisconsin-Madison \\
   \textbf{Program:} & Fuel Cycle R\&D (FC-5.1b) & &
\end{tabularx}



\vspace{0.2in}
\noindent\rule{\textwidth}{1pt}\\

\noindent\textbf{Abstract:} The goal of this
project is to build a framework dedicated to
uncertainty/error propagation for the Cyclus fuel
cycle simulator. In this project two main tasks
will be conducted. The first will allow Cyclus to
be aware of uncertainty by updating the material
class and all the corresponding operators
(ensuring the backward compatibility with other
Cyclus project). The second will introduce
uncertainty/error propagation in to Cyclus
archetypes.  In addition to improving existing
archetypes, new archetypes will be introduced that
are more responsive to incoming material
compisitions.  These archetypes will use neural
network approaches to build fuel according to
neutronics constraints and to determine the
evolution of the fuel during irradiation.

A variety of approaches for propagating
uncertainty and error will be explored, each
matched to the fidelity of the physics model of
the archetype in question.  For some, it is a
simple analytical propagation of error.  For
others, it may require neural netowrks, kriging or
other approaches.

The results will provide an important feature for
fuel cycle simulation. Given the appropriate
estimation of the error relative to any fuel cycle
simulation, the simulator would be able to make
decisions about fuel cycle transition like fuel
reprocessing, or the launching of new technologies
or types of reactors. Furthermore, the project
will be one of the first of its kind to
introducing error propagation in fuel cycle
calculation, increasing the utility of the Cyclus
kernel.  It is expected that this estimation of
uncertainty will help identify specific
sensitivities and also increase overall confidence
in the results of such simulations.
\end{document}
