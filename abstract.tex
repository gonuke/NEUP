\documentclass[dvips,12pt]{article}

% Any percent sign marks a comment to the end of the line

% Every latex document starts with a documentclass declaration like this
% The option dvips allows for graphics, 12pt is the font size, and article
%   is the style

\usepackage[letterpaper, margin=1in]{geometry}

\usepackage[pdftex]{graphicx}
\usepackage{tabularx}

\usepackage{fancyhdr}
\pagestyle{fancy}
\fancyhf{}
\chead{\includegraphics[width=2in]{neup_logo.png}}
\renewcommand{\headrulewidth}{0pt}

\newcommand{\unc}[1]
{ \delta #1 }

\newcommand{\uncsq}[1]
{ \left(\unc{#1}\right)^2 }

\newcommand{\uncratio}[1]
{ \left(\frac{\unc{#1}}{#1}\right) }

\newcommand{\uncratiosq}[1]
{ \uncratio{#1}^2 }

\newcommand{\uncvector}[1]
{ \left[ \begin{array}{c} #1 \\ \delta #1 \end{array} \right] }

\newcommand{\comment}[1]
{{\bfseries \color{red} #1}}


%----------------------------------------------------------------------------------------
%	DOCUMENT INFORMATION
%----------------------------------------------------------------------------------------
\begin{document}
\vspace*{0.5in}
\begin{centering}
  \textbf{\large Error and Uncertainty Propagation for Fuel Cycle Calculations}\\
\end{centering}

\noindent\rule{\textwidth}{0.4pt}\\

\noindent
\begin{tabular}{rll}
\textbf{PI:} & Paul Wilson & University of Wisconsin-Madison\\
\textbf{Collaborators:} & Baptiste Mouginot & University of Wisconsin-Madison\\
\textbf{Technical Workscope:} & Fuel Cycle  & (FC-5.1b)\\
\end{tabular}

\vspace{0.2in}
\noindent\rule{\textwidth}{0.4pt}\\

\noindent\textbf{Abstract:} 
The goal of this project is to build a framework
dedicated to uncertainty/error propagation for the
CYCLUS fuel cycle simulator. In this project two
main works will be conducted. The first will allow
CYCLUS to be uncertainty aware updating the
material basic class and all the corresponding
operator (insuring the backward compatibility with
other CYCLUS project). The second will be
dedicated to introduce incertainty/error to (and
improve) existing models providing algortyms to
build fuel according to neutronics constrains and
to determine the evolution of the fuel during
irradiation.

Error propagation could be a very complicated
matter and long to compute. Therefore to limite
the impact on the user experience we are
considering mainly a empirical approach to assess
and estimate the error/uncertainty impact of the
different modules. An empirical approach has the
advantage to concentrate the computationnal effort
in the development phase and then to be faster
during the usage. The models we are will developp
are based on pre-trained neural network prediction
(or Multilayer perceptron depending on naming
convention). This this proposal this proposal will
leverage existing neural network library
frameworks. Frameworks to be considered for
inclusion are TMVA (ROOT/CERN), FANN, neuralnet
from R or the pybrain.

The results will provide an important feature
needed in the fuel cycle simulator. Given the
appropriate estimation of the error relative to
any fuel cycle simulation, the simulator would be
able to make decisions about fuel cycle transition
like fuel reprocessing, or the launching of new
technologies or types of reactors. Furthermore,
the project will be one of the first of its kind
to introducing error propagation in fuel cycle
calculation, increasing the utility of the Cyclus
kernel.
\end{document}
