
\section{Motivation}
Nuclear fuel cycle simulation (NFCS) tools have a large scope of applications,
from the study of the behavior of a specific fuel type or reactor inside an
existing nuclear fleet to the prospective analysis of a complete transition to
advanced nuclear fuel cycle technologies. Unlike many simulation domains,
rigorous validation against experimental results is not viable for this kind of
simulation tool that incorporates aspects of human socio-political decision
making.  Nevertheless, it is important to establish confidence in the quality of
the results that are generated, with different degrees of confidence depending
on the use case.  The only existing way to develop confidence in any NFCS tool
is to compare with other similar tools or with simplified sets of historical
data. For the former, the conclusion of such a comparison is often a list of why
each software tool gives different results, but lacking quantitative information
on the precision of any of the results. The latter only allows validation of
existing concepts and has limited applicability to new fuel or reactor concepts
with varying technology readiness levels (TRLs).

There are a number of different tools used around the world to support decisions
on nuclear fuel cycle options.  While some tools offer a rigorous treatment of
the physics, they often require an experienced analyst to compute the results.
The risk is to offer more modeling precisions than is warranted given the
uncertainty in the inputs and related socio-political constraints. It might lead
to an over-complicated simulation with a very complex analysis, and a false
sense of certainty in the results. Furthermore, tool development/design
philosophy can potentially influence simulation outcome of a fuel cycle
analysis.

The aim of this project is to help the nuclear fuel cycle community to build
knowledge and understanding about the algorithms and models used by the
different fuel cycle simulation tools. Such knowledge could not only help fuel
cycle annalists to perform more pertinent and relevant study, but also convince
stakeholders to rely more on nuclear fuel cycle study to make nuclear decisions.


This project proposes to tackle those issues from both sides. The first part of
this project will be dedicated to create a framework allowing fuel cycle
analysts to understand how specific modeling choices or tool designs affect the
simulation outcomes. This framework will include a variety of problems defined
to analyse a set of features and their range of implementations in the different
NFCS tools. It will allow a collaborative definition of benchmark problems, the
evaluation of simulation outputs for these problems, and the cataloging of each
participant’s contribution.
The second part of this project will focus on linking modeling capability and
real study needs, aiming to assess the impact of modeling precision across
varying facilities in the nuclear fuel cycle in order to determine the
robustness of the conclusions to those variations in modeling precision. 


\section{Verification and Validation in Nuclear Fuel Cycle Tools}

Software verification and validation are two independent processes allowing respectively
to check that the software behaves as it is suppose to, and to check that its
responses it close to the reality, i.e. is it computing the right thing versus is it
computing the thing right.

While software verification can be achieve, using proper software testing
methods such as testing unit testing, validation in nuclear fuel cycle simulation is
almost impossible. 
Indeed, NFC tools aims to simulate, the historic or prospective, large nuclear fuel
cycle of an entity (company, region, country,\ldots) . For obvious
military and/or industrial strategic reasons most of the data about existing
nuclear fuel cycle is only partially publicly available, if available. Moreover,
there is no data available to validate prospective use of the NFC tools including
future facility concepts.

This is why several inter-comparison studies between NFC tools have been
performed. Because of their format and the wide bestiary of NFC
tools, those studies had a relatively small impact on the NFC community. Indeed
those studies 



generally did not take into account the 




