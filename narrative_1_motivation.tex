
\section{Motivation}
Nuclear fuel cycle simulation (NFCS) tools have a large scope of applications,
from the study of the behavior of a specific fuel type or reactor inside an
existing nuclear fleet to the prospective analysis of a complete transition to
advanced nuclear fuel cycle technologies. Unlike many simulation domains,
rigorous validation against experimental results is not viable for this kind of
simulation tool that incorporates aspects of human socio-political decision
making.  Nevertheless, it is important to establish confidence in the quality of
the results that are generated, with different degrees of confidence depending
on the use case.  The only existing way to develop confidence in any NFCS tool
is to compare with other similar tools or with simplified sets of historical
data. For the former, the conclusion of such a comparison is often a list of why
each software tool gives different results, but lacking quantitative information
on the precision of any of the results. The latter only allows validation of
existing concepts and has limited applicability to new fuel or reactor concepts
with varying technology readiness levels (TRLs).

There are a number of different tools used around the world to support decisions
on nuclear fuel cycle options.  While some tools offer a rigorous treatment of
the physics, they often require an experienced analyst to compute the results.
The risk is to offer more modeling precisions than is warranted given the
uncertainty in the inputs and related socio-political constraints. It might lead
to an over-complicated simulation with a very complex analysis, and a false
sense of certainty in the results. Furthermore, tool development/design
philosophy can potentially influence simulation outcome of a fuel cycle
analysis.
\section{State of the Art}

