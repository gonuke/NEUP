\documentclass[dvips,12pt]{article}

% Any percent sign marks a comment to the end of the line

% Every latex document starts with a documentclass declaration like this The
% option dvips allows for graphics, 12pt is the font size, and article is the
% style
\usepackage[letterpaper, margin=1in]{geometry}


\usepackage[pdftex]{graphicx}
\usepackage{url}
\usepackage[pdftex]{xcolor}
\usepackage{amsmath}
\usepackage{enumitem}
\usepackage{tabularx}
\usepackage{todonotes}

\usepackage[acronym]{glossaries}
\newacronym{NFCS}{NFCS}{nuclear fuel cycle simulator}
\newacronym{TRL}{TRL}{technology readiness level}
\newacronym{FCO}{FCO}{Fuel Cycle Options}
\newacronym{USDOE}{USDOE}{U.S. Department of Energy}
\newacronym{VnV}{V\&V}{verification and validation}
\newacronym{FCCI}{FCCI}{Fuel Cycle Confidence Initiative}
\newacronym{CHTC}{CHTC}{Center for High-Throughput Computing}
\newacronym{ACI}{ACI}{Advanced Computing Initiative}
\newacronym{OSG}{OSG}{Open Science Grid}
\newacronym{HTC}{HTC}{high-throughput computing}
\newacronym{HPC}{HPC}{high-performance computing}


\usepackage{fancyhdr}
\pagestyle{fancy}
\fancyhf{}
\lfoot{2018 CFA Narrative xxxxxxxx}
\rfoot{Page \thepage\ of \pageref{LastPage}}

\renewcommand{\headrulewidth}{0pt}
\renewcommand{\footrulewidth}{0.5pt}

\newcommand{\unc}[1]
{ \delta #1 }

\newcommand{\uncsq}[1]
{ \left(\unc{#1}\right)^2 }

\newcommand{\uncratio}[1]
{ \left(\frac{\unc{#1}}{#1}\right) }

\newcommand{\uncratiosq}[1]
{ \uncratio{#1}^2 }

\newcommand{\uncvector}[1]
{ \left[ \begin{array}{c} #1 \\ \delta #1 \end{array} \right] }

\newcommand{\comment}[1]
{{\bfseries \color{red} #1}}

\makeatletter
\renewcommand\section{\@startsection {section}{1}{\z@}%
                                   {-2.0ex \@plus -1ex \@minus -.2ex}%
                                   {2.3ex \@plus.2ex}%
                                   {\normalfont\bfseries}}% from \Large
\renewcommand\subsection{\@startsection{subsection}{2}{\z@}%
                                     {-2.0ex\@plus -1ex \@minus -.2ex}%
                                     {1.5ex \@plus .2ex}%
                                     {\normalfont\bfseries}}% from \large
\renewcommand\subsubsection{\@startsection{subsubsection}{3}{\z@}%
                                     {-2.0ex\@plus -1ex \@minus -.2ex}%
                                     {1.5ex \@plus .2ex}%
                                     {\normalfont\bfseries}}% from \normalsize
\makeatother


%----------------------------------------------------------------------------------------
%DOCUMENT INFORMATION
%----------------------------------------------------------------------------------------
\begin{document}
\begin{centering} Application Narrative for
  DE-FOA-0001772\\
  \textbf{\large Developing an Approach to Verification and Validation\\ for Nuclear Fuel Cycle Simulation Tools}\\
  CFA-18-15645 Narrative\\
\end{centering}
%\tableofcontents

\vspace{1em}

\noindent\textbf{Technical Workscope Identifier:}  \hspace{1.5in}
\textbf{Time Frame:}

% Some context and general goal
\section{Project Objectives}

The objective of this project is to develop an open and collaborative approach
for establishing increased confidence in \gls{NFCS} tools. This approach will
directly address verification by publishing jointly-developed problems with
reference solutions.  It will indirectly address validation through a growing
set of publicly available solutions to those problems contributed by
developers of such tools.  Together, this approach will allow the nuclear
fuel cycle community to develop a more nuanced understanding of the results
produced by their tools and, in so doing, be better prepared to express
confidence in those results when communicating results both within and without
that community.

This kind of comparison will also build knowledge and understanding about the
differences between the various algorithms and models used by the many fuel
cycle simulation tools. In particular, it is important to understand the
circumstances under which the choice of algorithm and/or model influences the
conclusions that one might draw from such a simulation. Such knowledge could
not only help fuel cycle analysts to perform more pertinent and relevant
studies, but ultimately to convince stakeholders to rely more on nuclear fuel
cycle analysis in the decision making processes relative to nuclear
technology.

The development of this capability in an open and collaborative fashion will
support innovation in this space, as it makes it possible for new entrants to
immediately begin testing their tools against established solutions without
waiting for a coordinated benchmarking effort.  Such coordinated efforts have
occurred in the past, but their results are generally not published in a way
that invites comparison by future analysts who were not part of the initial
exercise.


\input{narrative_2_scope}

\section{Logical path to accomplishing scope}

This project has two primary tasks:
\begin{enumerate}
\item developing an open, collaborative benchmark framework allowing real
  inter-code comparison, and
\item using this framework to understand what modeling fidelity is required to
  ensure the robustness of conclusions for different fuel cycle scenarios.
\end{enumerate}


% Collaborative Benchmark
\input{task_1_fcci}

% Robustness part
\input{task_2_robustness}


%\input{narrative_4_
% Timeline - Milestone - Deliverables 
\input{narrative_5}

\section{Quality Assurance \& Software Development Process}

\textit{This project will comply with all Quality Assurance requirements, as
described on the NEUP website.}

This work will follow the Cyclus development model.  We will employ a variety of
modern software project management tools: revision control, automated testing,
automated and manual documentation, bug tracking. Related projects at are
already managed under source code revision control system (git) that provides
detailed tracking of all changes to the code base. A test suite has been
developed and new tests will be added for each new capability. This test suite
will be automatically executed with each proposed change in the revision control
system to identify any flaws that may be introduced. Automated documentation
tools are used in the source code to create a detailed reference for the
interfaces and additional background documentation. Out-of-code detailed reports
and publications will supplement the information on each new capability.
Finally, a bug tracking system (GitHub issues) is deployed to help users and
developers to understand known issues and to track their resolution as a
developer community. As new capability is added, it will come under the same
quality assurance practices as described here for the existing capabilities.

%----------------------------------------------------------------------------------------
%BIBLIOGRAPHY
%----------------------------------------------------------------------------------------

\bibliographystyle{narrative}

\bibliography{narrative}

%----------------------------------------------------------------------------------------

\label{LastPage} \end{document}
