\documentclass[dvips,11pt]{article}

% Any percent sign marks a comment to the end of the line

% Every latex document starts with a documentclass declaration like this The
% option dvips allows for graphics, 12pt is the font size, and article is the
% style
\usepackage[letterpaper, margin=1in]{geometry}

\usepackage[pdftex]{graphicx}
\usepackage{url}
\usepackage[pdftex]{xcolor}
\usepackage{amsmath}
\usepackage{enumitem}
\usepackage{tabularx}

\usepackage{fancyhdr}
\pagestyle{fancy}
\fancyhf{}
\lfoot{20XX CFA Narrative xxxxxxxx}
\rfoot{Page \thepage\ of \pageref{LastPage}}

\renewcommand{\headrulewidth}{0pt}
\renewcommand{\footrulewidth}{0.5pt}

\newcommand{\unc}[1]
{ \delta #1 }

\newcommand{\uncsq}[1]
{ \left(\unc{#1}\right)^2 }

\newcommand{\uncratio}[1]
{ \left(\frac{\unc{#1}}{#1}\right) }

\newcommand{\uncratiosq}[1]
{ \uncratio{#1}^2 }

\newcommand{\uncvector}[1]
{ \left[ \begin{array}{c} #1 \\ \delta #1 \end{array} \right] }

\newcommand{\comment}[1]
{{\bfseries \color{red} #1}}

\makeatletter
\renewcommand\section{\@startsection {section}{1}{\z@}%
                                   {-2.0ex \@plus -1ex \@minus -.2ex}%
                                   {2.3ex \@plus.2ex}%
                                   {\normalfont\bfseries}}% from \Large
\renewcommand\subsection{\@startsection{subsection}{2}{\z@}%
                                     {-2.0ex\@plus -1ex \@minus -.2ex}%
                                     {1.5ex \@plus .2ex}%
                                     {\normalfont\bfseries}}% from \large
\renewcommand\subsubsection{\@startsection{subsubsection}{3}{\z@}%
                                     {-2.0ex\@plus -1ex \@minus -.2ex}%
                                     {1.5ex \@plus .2ex}%
                                     {\normalfont\bfseries}}% from \normalsize
\makeatother


%----------------------------------------------------------------------------------------
%DOCUMENT INFORMATION
%----------------------------------------------------------------------------------------
\begin{document} 
\begin{centering} Application Narrative for
    DE-FOA-XXXXXXXXXXX\\ \textbf{\large }\\ 20XX CFA Narrative xxxxxxx\\
\end{centering} 

\vspace{1em}

\noindent\textbf{Technical Workscope Identifier:}  \hspace{1.5in}
\textbf{Time Frame:}

\section{Motivation}
Nuclear fuel cycle simulation (NFCS) tools have a large scope of applications,
from the study of the behavior of a specific fuel type or reactor inside an
existing nuclear fleet to the prospective analysis of a complete transition to
advanced nuclear fuel cycle technologies. Unlike many simulation domains,
rigorous validation against experimental results is not viable for this kind of
simulation tool that incorporates aspects of human socio-political decision
making.  Nevertheless, it is important to establish confidence in the quality of
the results that are generated, with different degrees of confidence depending
on the use case.  The only existing way to develop confidence in any NFCS tool
is to compare with other similar tools or with simplified sets of historical
data. For the former, the conclusion of such a comparison is often a list of why
each software tool gives different results, but lacking quantitative information
on the precision of any of the results. The latter only allows validation of
existing concepts and has limited applicability to new fuel or reactor concepts
with varying technology readiness levels (TRLs).

There are a number of different tools used around the world to support decisions
on nuclear fuel cycle options.  While some tools offer a rigorous treatment of
the physics, they often require an experienced analyst to compute the results.
The risk is to offer more modeling precisions than is warranted given the
uncertainty in the inputs and related socio-political constraints. It might lead
to an over-complicated simulation with a very complex analysis, and a false
sense of certainty in the results. Furthermore, tool development/design
philosophy can potentially influence simulation outcome of a fuel cycle
analysis.
\section{State of the Art}

\section{The project}

\subsection{FCCI}

\subsection{Robustness assessment}
The second art of this project will be dedicated to the
measurement/determination of the impact of different modeling choices on
different output metrics as a function of the studied scenario.  In the early
age of the nuclear fuel cycle simulation because of the limitation of the
computational power, the capabilities of the fuel cycle simulation were limited
and the subject to strong simplification. But, the refinement of the different
models did grows with the available computers power. 

The different actors of the fuel cycle community do not agree on the requirement
to use the highest modeling fidelity. While the primary intuition supports
always using the best modeling options, some experts argue that the different
hypothesis and unknown of a prospective scenario are largely dominate the
uncertainty on the output metrics; considering better modeling option is then a
waste of time and effort.

Nevertheless it is possible to pictures use cases where some higher level of
refinement are required to achieve correct or better conclusion. For example, a
simple description of the enrichment facility (a simple relation between the
different assay with a SWU constrain) will be enough for the simulation of the
historical a national fleet, a fuel cycle evaluation of the JCPOA agreement will
require a fine description of the enrichment cascade\dots
Moreover there is no scientific evaluation of the importance on the output
metrics uncertainty of the modeling precision over the input errors and
uncertainties.
The part of the project aims to evaluate the impact of the different modeling
choices on the output metrics and compare them to the one of the input metrics
uncertainties.


\section{Quality Assurance \& Software Development Process}

\textit{This project will comply with all Quality Assurance requirements, as
described on the NEUP website.}

This work will follow the Cyclus development model.  We will employ a variety of
modern software project management tools: revision control, automated testing,
automated and manual documentation, bug tracking. Related projects at are
already managed under source code revision control system (git) that provides
detailed tracking of all changes to the code base. A test suite has been
developed and new tests will be added for each new capability. This test suite
will be automatically executed with each proposed change in the revision control
system to identify any flaws that may be introduced. Automated documentation
tools are used in the source code to create a detailed reference for the
interfaces and additional background documentation. Out-of-code detailed reports
and publications will supplement the information on each new capability.
Finally, a bug tracking system (GitHub issues) is deployed to help users and
developers to understand known issues and to track their resolution as a
developer community. As new capability is added, it will come under the same
quality assurance practices as described here for the existing capabilities.

%----------------------------------------------------------------------------------------
%BIBLIOGRAPHY
%----------------------------------------------------------------------------------------

\bibliographystyle{narrative}

\bibliography{narrative}

%----------------------------------------------------------------------------------------

\label{LastPage} \end{document}
