\section{Project Objectives}

The objective of this project is to develop an open and collaborative approach
for establishing increased confidence in \gls{NFCS} tools. This approach will
directly address verification by publishing jointly-developed problems with
reference solutions.  It will indirectly address validation through a growing
set of publicly available solutions to those problems contributed by
developers of such tools.  Together, this approach will allow the nuclear
fuel cycle community to develop a more nuanced understanding of the results
produced by their tools and, in so doing, be better prepared to express
confidence in those results when communicating results both within and without
that community.

This kind of comparison will also build knowledge and understanding about the
differences between the various algorithms and models used by the many
NFCS tools. In particular, it is important to understand the
circumstances under which the choice of algorithm and/or model influences the
conclusions that one might draw from such a simulation. Such knowledge could
not only help fuel cycle analysts to perform more pertinent and relevant
studies, but ultimately to convince stakeholders to rely more on nuclear fuel
cycle analysis in the decision making processes relative to nuclear
technology.

The development of this capability in an open and collaborative fashion will
support innovation in this space, as it makes possible for new entrants to
immediately begin testing their tools against established solutions without
waiting for a coordinated benchmarking effort.  Such coordinated efforts have
occurred in the past, but their results are generally not published in a way
that invites comparison by future analysts who were not part of the initial
exercise.

%% High Relevance: The project is fully supportive of, and has significant,
%% easily recognized and demonstrable ties to mission and relevant workscope
%% area. The project builds on synergies with ongoing direct- or
%% competitively-funded projects or meets a critical mission need. The project
%% focuses on critical knowledge gaps where limited work is currently being
%% performed.

This project is relevant to the long term mission of the Office of Nuclear
Energy as it looks to rely on \gls{NFCS} tools to assist with long term planning
of research and development programs.  The recent \gls{USDOE} \gls{FCO}
campaign demonstrated how this kind of analysis might be used to better
understand the transitions that would need to occur over the many-decade time
scale to arrive at demonstrably improved nuclear fuel cycles.  The fidelity
%I don't understand the last part ``at demonstrably improved nuclear fuel
%cycles''
and decision spaces of those analyses is almost certainly insufficient to
understand the trajectory of such a transition that must be sustained over
time scales that can incorporate substantial socio-economic transience.
Developing confidence in these tools prior to the need to use them to support
important decision-making is wise and probably fiscally prudent.  Many
existing tools inherently presume centralized decision making in planning long
term nuclear futures even though domestic experience is likely to develop more
organically, driven by market forces and possible influenced by centralized
policy-making.  This will provide a platform to study that inconsistency, in
addition to the technical variations that exist among different \gls{NFCS}
tools.

%% Review Criterion #1
%% Advances the State of Scientific Knowledge and Understanding and Addresses
%% Gaps in Nuclear Science and Engineering Research: The technical merit of the
%% proposed R&D project will be evaluated, including the extent to which the
%% project advances the state of scientific knowledge and understanding and
%% addresses gaps in nuclear science and engineering research. Evaluation will
%% consider how important the proposed project is to advancing knowledge and
%% understanding within the area selected and how well the proposed project
%% advances, discovers, or explores creative, original, or potentially
%% transformative concepts.

This project aims to deliver a near approximation to validation, for the first
time, in a simulation space that has so far eluded rigorous validation
efforts.  This will advance our knowledge and understanding of \gls{NFCS}['s]
and their role in supporting analysis of advanced nuclear fuel cycles and
decision-making around future directions in our nuclear fuel cycles.
Substantial resources have been invested in a variety of \gls{NFCS} tools,
whether to serve specific purposes, or as they expanded to more general
purpose, but confidence in each new analysis remains fragile because of a lack
of generic ability to develop confidence in such tools.
