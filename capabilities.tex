\documentclass[dvips,12pt]{article}

% Any percent sign marks a comment to the end of the line

% Every latex document starts with a documentclass declaration like this
% The option dvips allows for graphics, 12pt is the font size, and article
%   is the style

\usepackage[letterpaper, margin=1in]{geometry}

\usepackage[pdftex]{graphicx}
\usepackage{url}
\usepackage[pdftex]{xcolor}
\usepackage{amsmath}
\usepackage{enumitem}
\usepackage{tabularx}
\usepackage[acronyms]{glossaries}
\newacronym{NFCS}{NFCS}{nuclear fuel cycle simulator}
\newacronym{TRL}{TRL}{technology readiness level}
\newacronym{FCO}{FCO}{Fuel Cycle Options}
\newacronym{USDOE}{USDOE}{U.S. Department of Energy}
\newacronym{VnV}{V\&V}{verification and validation}
\newacronym{FCCI}{FCCI}{Fuel Cycle Confidence Initiative}
\newacronym{CHTC}{CHTC}{Center for High-Throughput Computing}
\newacronym{ACI}{ACI}{Advanced Computing Initiative}
\newacronym{OSG}{OSG}{Open Science Grid}
\newacronym{HTC}{HTC}{high-throughput computing}
\newacronym{HPC}{HPC}{high-performance computing}

\input{include/defs}


\usepackage{fancyhdr}
\pagestyle{fancy}
\fancyhf{}
\lfoot{2018 CFA Capabilities \ID}
\rfoot{Page \thepage\ of \pageref{LastPage}}

\renewcommand{\headrulewidth}{0pt}
\renewcommand{\footrulewidth}{0.5pt}

\newcommand{\unc}[1]
{ \delta #1 }

\newcommand{\uncsq}[1]
{ \left(\unc{#1}\right)^2 }

\newcommand{\uncratio}[1]
{ \left(\frac{\unc{#1}}{#1}\right) }

\newcommand{\uncratiosq}[1]
{ \uncratio{#1}^2 }

\newcommand{\uncvector}[1]
{ \left[ \begin{array}{c} #1 \\ \delta #1 \end{array} \right] }

\newcommand{\comment}[1]
{{\bfseries \color{red} #1}}

\makeatletter
\renewcommand\section{\@startsection {section}{1}{\z@}%
                                   {-2.0ex \@plus -1ex \@minus -.2ex}%
                                   {2.3ex \@plus.2ex}%
                                   {\normalfont\bfseries}}% from \Large
\renewcommand\subsection{\@startsection{subsection}{2}{\z@}%
                                     {-2.0ex\@plus -1ex \@minus -.2ex}%
                                     {1.5ex \@plus .2ex}%
                                     {\normalfont\bfseries}}% from \large
\renewcommand\subsubsection{\@startsection{subsubsection}{3}{\z@}%
                                     {-2.0ex\@plus -1ex \@minus -.2ex}%
                                     {1.5ex \@plus .2ex}%
                                     {\normalfont\bfseries}}% from \normalsize
\makeatother


%----------------------------------------------------------------------------------------
%	DOCUMENT INFORMATION
%----------------------------------------------------------------------------------------
\begin{document}
\begin{centering}
  Application Narrative for DE-FOA-0001772\\
  \textbf{\large \mytitle}\\
  2018 CFA Capabilities \ID\\
\end{centering}
\vspace{1em}

\noindent\textbf{Technical Workscope Identifier:} \workscope \hspace{0.5in}
\textbf{Time Frame:} \timeframe\\

The UW-Madison has access to a variety of computing resources and services
that facilitate the successful completion of this project.  The primary
development work will be carried out on a desktop-class computer, provided by
this project, and fully connected to the UW-Madison's network providing high
bandwidth access to a variety of on- and off-campus resources.  When
computational needs require larger resources, existing UW-Madison technology
infrastructure supported by the \gls{CHTC} can be readily leveraged, including
CPU capacity, network connectivity, storage availability, and middleware
connectivity. The UW-Madison's \gls{ACI} has invested in the \gls{CHTC} as the
primary provider of shared large-scale computing infrastructure to campus
researchers, and all standard \gls{CHTC} services are provided free-of-charge
to campus researchers. For \gls{HTC} capability,
UW-Madison maintains many compute clusters across campus, which are managed
via software developed by the UW-Madison's HTCondor Project distributed
computing research group; therefore, these clusters are linked together to
share resources via widely adopted distributed computing
technologies. Together these clusters represent roughly 30,000 CPU cores in
support of research. Between 7/1/2016 and 6/31/2017, the \gls{CHTC} supported
more than 380 million CPU hours of computing work. Should these resources not
be sufficient for your project, \gls{CHTC} users can also engage computing
resources from the \gls{OSG}, also at no cost to
researchers. For \gls{HPC} capability, the \gls{CHTC}
also maintains a shared-use cluster of roughly 7000 tightly coupled cores,
with expansion room to increase cluster compute capacity as campus needs grow
and as research groups contribute for dedicated access to a number of
cores. Compute nodes have 16 or 20 cores, each, and 64 or 128 GB RAM, with
access to a shared file system and resources managed via the open-source
software, SLURM.  Between the ACI, \gls{CHTC}, and aforementioned HTCondor
Project, UW-Madison is home to over 20 full-time staff with a proven track
record of making compute middleware work for scientists. Far beyond just being
familiar with the deployment and use of such software, UW staff has been
intimately involved in its design and implementation. Furthermore, the
\gls{CHTC} provides consulting for the development of robust HTC and HPC
research methods, support for grant proposal development, and a variety of
formal and informal training opportunities for users of \gls{CHTC} resources.




\label{LastPage}
\end{document}
