\documentclass[dvips,12pt]{article}

% Any percent sign marks a comment to the end of the line

% Every latex document starts with a documentclass declaration like this
% The option dvips allows for graphics, 12pt is the font size, and article
%   is the style

\usepackage[letterpaper, margin=1in]{geometry}

\usepackage[pdftex]{graphicx}
\usepackage{url}
\usepackage[pdftex]{xcolor}
\usepackage{amsmath}
\usepackage{enumitem}
\usepackage{tabularx}

\usepackage{fancyhdr}
\pagestyle{fancy}
\fancyhf{}
\lfoot{2016 CFA Capabilities 10565}
\rfoot{Page \thepage\ of \pageref{LastPage}}

\renewcommand{\headrulewidth}{0pt}
\renewcommand{\footrulewidth}{0.5pt}

\newcommand{\unc}[1]
{ \delta #1 }

\newcommand{\uncsq}[1]
{ \left(\unc{#1}\right)^2 }

\newcommand{\uncratio}[1]
{ \left(\frac{\unc{#1}}{#1}\right) }

\newcommand{\uncratiosq}[1]
{ \uncratio{#1}^2 }

\newcommand{\uncvector}[1]
{ \left[ \begin{array}{c} #1 \\ \delta #1 \end{array} \right] }

\newcommand{\comment}[1]
{{\bfseries \color{red} #1}}

\makeatletter
\renewcommand\section{\@startsection {section}{1}{\z@}%
                                   {-2.0ex \@plus -1ex \@minus -.2ex}%
                                   {2.3ex \@plus.2ex}%
                                   {\normalfont\bfseries}}% from \Large
\renewcommand\subsection{\@startsection{subsection}{2}{\z@}%
                                     {-2.0ex\@plus -1ex \@minus -.2ex}%
                                     {1.5ex \@plus .2ex}%
                                     {\normalfont\bfseries}}% from \large
\renewcommand\subsubsection{\@startsection{subsubsection}{3}{\z@}%
                                     {-2.0ex\@plus -1ex \@minus -.2ex}%
                                     {1.5ex \@plus .2ex}%
                                     {\normalfont\bfseries}}% from \normalsize
\makeatother


%----------------------------------------------------------------------------------------
%	DOCUMENT INFORMATION
%----------------------------------------------------------------------------------------
\begin{document}
\begin{centering}
  Application Narrative for DE-FOA-0001281\\
  \textbf{\large Error and Uncertainty Propagation for Fuel Cycle Calculations}\\
  2016 CFA Capabilities 10565\\
\end{centering}
\vspace{1em}

\noindent\textbf{Technical Workscope Identifier:} FC-5.1b \hspace{1in} \textbf{Time Frame:} 3 years\\

Most of the development work for this project will be carried out on
desktop workstation computers with free and open source software and
software libraries.

For performing large parametric analyses, the University of Wisconsin
has a shared computing capability with a combination of high
throughput computing (HTC) and high performance computing (HPC)
resources.  The UW-Madison’s Center for High Throughput Computing
(CHTC) provides open access to these computing facilities for all
researchers at no cost, with options for purchasing preferred access
if the researcher’s needs warrant it.

The CHTC manages a distributed computing resource with approximately
5,000 dedicated cores and 5,000 opportunistic cores, and provides
seamless access to the Open Science Grid and other national resources.
This HTC capability routinely provides 300,000 CPU hours per day to UW
researchers and is well suited to the calculation modes expected in
this proposal.  This Center is also exploring opportunities to offer
free or low cost access to commercial computing resources, including
Amazon EC2.

The CHTC also manages a new high performance computing (HPC) cluster,
with 768 cores, 3 TB of RAM, and a 56 Gb/s Infiniband interconnect.
This system is projected to grow over the next three years to up to 4
times this size.

The Cyclus team maintains a modern software development infrastructure
in support of ongoing software quality assurance.  This infrastructure
includes a variety of modern software project management tools:
revision control, automated testing, automated and manual
documentation, bug tracking. Related projects are already managed
under source code revision control system (git/GitHub) that provides
detailed tracking of all changes to the code base. A test suite has
been developed and new tests will be added for each new
capability. This test suite is automatically executed (CircleCI) with
each proposed change in the revision control system to identify any
flaws that may be introduced. Automated documentation tools are used
in the source code to create a detailed reference for the interfaces
and additional background documentation. Out-of-code detailed reports
and publications will supplement the information on each new
capability. Finally, a bug tracking system (GitHub issues) is deployed
to help users and developers to understand known issues and to track
their resolution as a developer community. As new capability is added,
it will come under the same quality assurance practices as described
here for the existing capabilities.

\label{LastPage}
\end{document}
